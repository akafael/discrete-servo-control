% Pacotes e configurações padrão do estilo ``article''\
% -------------------------------------
\documentclass[a4paper,11pt]{article}
% Layout
% ------------------------------------------------------------------------------
\input{relat_layout.tex}

%\usepackage{circuitikz}

\title{Proposta de projeto - Controle discreto de um motor DC} % Define o título do Relatório
\author{Rafael Lima}

% Definições Auxiliares ( Macros próprias )
% ------------------------------------------------------------------------------
%\input{relat_aux.tex} % Arquivo com minhas macros
% ----------------------------------~>ø<~---------------------------------------
\begin{document}
% Capa e Índice ----------------------------------------------------------------
%--------------------------------------------------- Capa --------------------------------------------
%\newpage
\begin{figure}[h!]
\centering
\includegraphics[scale=0.9]{img/simb_unb.png}
\label{fig:unb}
\end{figure}

\begin{center}
{\LARGE Universidade de Brasília}\\
Departamento de Engenharia Elétrica\\
Professor: Henrique Cezar Ferreira\\
Disciplina: Controle Digital\\
\end{center}


\vspace{0.18\textheight}

\begin{center}
    \Huge \textbf{\\\thetitle \\}
\end{center}

\vspace*{\fill} % Completa espaço em branco e empurra o resto para o final da página

% Tabela com os nome das pessoas do grupo

\begin{table}[H]
    \begin{tabular}{ll}
        % Nome      & Matrícula
        Rafael Lima & 10/0131093 \\
        Leonardo Cardoso Botelho & 11/0154151 \\
    \end{tabular}
\end{table}

\vspace{0.5cm}

\begin{center}
    \textbf{Brasília\\
    \the\year} % Coloca o Ano atual
\end{center}

\thispagestyle{empty} % Retira o cabeçalho e o rodapé da página

% ------------------------------------------------- Índice -------------------------------------------
\newpage
\tableofcontents
\newpage
% ----------------------------------------------------------------------------------------------------

 % Capa para UnB
% Conteúdo ---------------------------------------------------------------------

\section{Introdução}

A presente proposta de projeto propõe o desenvolvimento de um sistema de controle discreto para uma planta composta por um motor DC (motor de corrente contínua), um sistema de redução e um sensor encoder de maneira a permitir o controle da posição a partir de um sinal de referência.

O motor de corrente contínua é amplamente utilizado em projetos de eletrônica e sua modelagem também está muito bem descrita na literatura de controle em geral. 

Pretende-se aplicar ao fim do desenvolvimento o controlador projetado em uma planta real. O conjunto de peças foi retirada de um equipamento antigo que havia sido descartado. Para implementação do controle, além da planta será utilizado um Arduino UNO e uma módulo com uma ponte H para o controle de potência conforme mostrado na figura \ref{fig:dispositivos}.

\begin{figure}[H]
    \centering
    \begin{subfigure}[b]{0.32\linewidth}
        \centering
        \includegraphics[width=0.8\linewidth]{src/tex/img/servomotor.jpg}
        \caption{Conjunto Motor DC + Encoder}
    \end{subfigure}
    \hfill
    \begin{subfigure}[b]{0.32\linewidth}
        \centering
        \includegraphics[height=0.9\linewidth, angle=90]{src/tex/img/arduinoUNO.jpg}
        \caption{Arduino UNO}
    \end{subfigure}
    \hfill
    \begin{subfigure}[b]{0.32\linewidth}
        \centering
        \includegraphics[width=0.9\linewidth]{src/tex/img/ponteH.jpg}
        \caption{Ponte H}
    \end{subfigure}
    \caption{Peças para proposta da planta}
    \label{fig:dispositivos}
\end{figure}

\section{Justificativa}

Sistemas de posicionamento utilizando motores de corrente contínua está presente nos mais variados equipamentos presentes nos dia de hoje. Isso se dá pois eles possuem bom custo benefício, sua implementação e manutenção são facilitadas.

Sua complexidade é razoável para o contexto da disciplina de Controle de Digital pois temos nesse sistema pelo menos dois atrasos nos elementos de atuação e sensoreamento. A caixa de redução por sua vez também insere um aspecto interessante ao processo, visto os efeitos de histerese e zona morta.

Do ponto de vista didático, esse sistema possui bons elementos a serem compreendidos no contexto da disciplina de Controle Digital. Podemos destacar as etapas de modelagem, discretização da planta, alocação de polos no domínio Z, projeto em LGR no domínio discreto de um controlador, identificação de sistema e implementação em uma planta real.

Por fim, o controle digital pressupõe um sistema computacional para sua execução, podendo ser operado por um microcontrolador disponível a custo acessível comercialmente. Isso torna esse projeto relevante no contexto das aplicações que utilizam motores de corrente contínua.


\section{Objetivos}

Propõe-se a obter nesse projeto os seguintes objetivos:

 \begin{enumerate}
   \item Obter modelo da planta no domínio da frequência (Motor DC em conjunto com caixa de redução);
   \item Obter modelo do sensor no domínio da frequência;
   \item Obter modelos discreto da Planta e do sensor
   \item Desenvolver modelo em \textit{Simulink} para o projeto
   \item Obter controlado descrito através de equações de diferenças;
   \item Implementação e Validação do controlador em teste em um sistema real;
 \end{enumerate}

\section{Metodologia de Projeto}

Para o desenvolvimento do projeto foram propostos as etapas relatadas a seguir. Primeiramente será uma revisão bibliográfica e modelagem da planta e avaliação em ambiente de simulação. Num segunda fase será ajustado o modelo encontrado para o planta real.

\begin{itemize}
    \item Estudo Planta Simulada
    \begin{enumerate}
        \item Revisão Bibliográfica para busca do modelo de um servo motor na literatura
        \item Simulação do modelo
        \item Implementação do controle em simulação
    \end{enumerate}
    \item Estudo Planta Real
    \begin{enumerate}
        \item Simulação da planta no \textit{Tinkercad}
        \item Identificação de Parâmetro da Planta
        \item Simulação da planta a partir do modelo identificado
        \item Implementação do controle em simulação
        \item Implementação do controle na planta
    \end{enumerate}
\end{itemize}

\section{Desenvolvimento}

\subsection{Implementação em Hardware}

Para garantir os requisitos necessários para a implementação do controlador em tempo discreto foi necessário implementar algumas funcionalidades do zero. Inicialmente tudo foi implementado no Tinkercad e posteriormente foi adotado o estudo direto na planta.

\subsubsection{Simulação no Tinkercad}

Como forma de prevenir danos as peças, foi utilizado o Tinkercad para simulação do sistema e como uma plataforma de estudos a respeito do comportamento do motor DC em conjunto com o encoder incremental. O Tinkercad foi escolhido por sua simplicidade de uso e facilidade de integração do circuito diretamente com a programação do Arduino.

\begin{figure}[H]
    \centering
    \includegraphics[width=\linewidth]{src/tex/img/pid_tinkercad.png}
    \caption{Diagrama Sistema implementado no Tinkercad}
    \label{fig:pid_tinkercad}
\end{figure}

Desta forma foi possível avaliar os algoritmos usados para a contagem de passo pelo encoder em conjunto com o acionamento do motor, cálculo do PID e detecção de sentido de rotação. Conforme ilustrado pela figura \ref{fig:pid_tinkercad}, foram utilizados potenciômetros para um ajuste manual e grosseiro dos ganhos do PID. Também foi adotado um display para mostrar os ganhos adotados. Com isto foi possível chegar a uma combinação de ganhos que permitia o motor alcançar o valor da referência ou ao menos parar após um tempo de movimento.

Dado as limitações da plataforma, principalmente nos modelos adotados para o motor e para comportamento das portas e ações do Arduíno não foi possível adotar nenhuma forma de identificação dos parâmetros. Concluído o estudo preliminar passamos para experimentos feitos diretamente na planta real.

\subsubsection{Temporização}

Como o estilo de controlador adotado pressupõe uma amostragem fixa, foi necessário adotar um controle de temporização estrito como forma de garantir que tanto o acionamento como a leitura do encoder seria feita dentro do tempo adequado. A leitura de cada passo do encoder é feita através de interrupção e a frequência de acionamento é feita a partir da comparação do valor de tempo para o instante atual com temporizadores internos.

\subsubsection{Leitura Encoder}

O encoder presente na planta escolhida é sistema ótico e incremental e com isto foi necessário adicionar ao código no Arduíno a contagem de passos a detecção de sentido de rotação. O sinal do encoder é composto por duas ondas quadradas defasadas em 90 graus. Desta forma, o deslocamento absoluto pode ser obtido através da contagem dos pulsos e o sentido de rotação pela comparação do dois sinais.

% TODO Adicionar Ilustração do sinal do encoder

Como esta variação é relativamente rápida, o sinal de borda de subida do encoder foi associado a uma interrupção no processador. Como efeito a cada vez que é detectado um pulso, o processamento é interrompido e uma função para contagem e detecção da direção é chamada. Desta forma é possível manter o algoritmo de controle em paralelo com a leitura do encoder.

A partir da captura de dados movendo-se manualmente o eixo do motor foi detectado a presença de bastante ruído na leitura. Foi percebido nos primeiros experimentos que contador acabava sendo incrementado ainda que o motor estivesse parado. Para reduzir o ruído algumas medidas foram adotadas: primeiro o sinal de alimentação do encoder foi separado do sinal de alimentação da ponte H, foi adotado um cabo com par trançado como forma de reduzir a interferência devido ao ruído introduzido com o acionamento do motor, a fonte de alimentação do motor e do arduino também foi separada. Com estas medidas, o ruído reduziu bastante.

\subsubsection{Acionamento Motor}

O acionamento do motor foi feito através de uma ponte H. Um fator que introduz não linearidade ao sistema é a presença de uma zona morta correspondente ao mínimo de tensão necessária para fazer o motor girar partindo do repouso. Além disto o sistema possui uma limitação de tensão máxima que é capaz de fornecer, o que também impõe um limite a região linear para controle do motor.

\subsection{Caracterização Sistema}

\begin{figure}[H]
    \centering
    \includegraphics[width=\linewidth]{src/tex/img/full_system.jpg}
    \caption{Planta Montada}
    \label{fig:pid_tinkercad}
\end{figure}

\subsubsection{Região Não Linear}

Idealmente o comportamento poderia ser representado através de um modelo linear para todas as condições de operação, No entanto, na prática devido as características do hardware, temos uma região de operação a qual o modelo pode ser utilizado delimitada por uma região inferior e superior.

Para o motor começar a se mover é preciso fornecer um mínimo de energia. Esta quantidade varia de acordo com as características eletromecânicas do motor. Em particular para o motor utilizado, a faixa de operação é de 3V a 6V, sendo 3V o mínimo para fazer o motor girar e 6V uma tensão máxima para não danificar.

A conversão do sinal de controle do Arduino em um sinal de potência foi feita através uma ponte H em conjunto com um sinal PWM. Como efeito a tensão de alimentação da fonte utilizada para o motor foi amostrada usando 8 bits. Desta forma, chegar ao valor teórico correspondente a zona morta do motor para uma alimentação de 5V basta 

$$
D_{min} = \frac{3V}{V_{fonte}}(2^8 -1) = \frac{3*255}{5} = 153
$$

Como forma de obter este valor experimentalmente foi feito um programa aplicando um sinal de entrada de dente de serra. Isto é o valor de referência passado para a tensão do motor foi gradualmente aumentado e em seguida reduzido. Os valores obtidos foram registrados a partir do Matlab através da conexão Serial com o Arduino.

Mesmo que frequência interna do Arduino seja próximo de $14 MHz$, as operações de comunicação via Serial são bastante dispendiosas. Por conta disto, o menor período conseguido experimentalmente a qual é mantida um a operação com ciclo em tempo contante foi de $60ms$. Para períodos menores que este valor, o tempo gasto com a comunicação serial, cálculos e as interrupções do encoder acabava sendo maior que o período e as garantias de funcionamento em tempo real eram perdidas.

\subsubsection{Identificação de Parâmetros}

Uma vez definido uma região linear de operação foi adotado o procedimento de identificação do sistema. Pelas especificação do fabricante temos

\begin{table}[H]
    $$
    \begin{array}{ccc}
         \hline
         V_i\ [V] & \dot{\theta}\ [RPM] & \dot{\theta}\ [Hz]\\
         \hline
         3 & 3000 & 50 \\
         6 & 6000 & 60 \\
         \hline
    \end{array}
    $$
    \caption{Velocidade de operação sem carga}
    \label{tab:dcmotor_speed}
\end{table}

De forma que poderíamos adotar como uma aproximação razoável do comportamento dinâmico do motor como uma função de transferência de primeira ordem na forma

\begin{equation}
    G_m(s) = \frac{\beta}{\alpha s + 1}
\end{equation}

Neste sistema temos apenas a medida da posição através do encoder. Este processo é feito através do conjunto de várias etapas, incluindo amostragem através da contagem do pulsos e a discretização referente ao instante de contabilização da leitura do encoder. No entanto, dado a grande diferença entre o tempo gasto neste processo com o tempo gasto na comunicação serial, podemos aproximar o sistema de medição de posição apenas como um ganho.

Feito as devidas considerações, a planta sem controlador pode ser representada da seguinte forma

%% TODO Acrescentar diagrama de blocos do sistema em malha aberta ( ZOH + Planta de 1 ordem + integrador  + ganho)

Aplicando a transformada Z podemos representar o sistema em tempo discreto pela seguinte função:

\begin{equation}\label{eq:gz_general}
    G(z) = \frac{Y(z)}{U(z)} = \frac{\beta_1 z + \beta_2}{z^2 + \alpha_1 z + \alpha_2}
\end{equation}

Do que podemos observar que a discretização traz como efeitos tanto o deslocamento dos polos, como também a inclusão de um zero na planta.

A partir da função \ref{eq:gz_general} podemos obter a representação do sistema a partir de equação de diferenças como:
$$
Y z^2
$$

\begin{equation}\label{eq:gz_general}
  Y[k+2] = -Y[k+1]\alpha_1 - Y[k]\alpha_2 + U[k+1]\beta_1 + U[k]\beta_2
\end{equation}

Tomando um conjunto maior de dados podemos reescrever a equação na forma matricial como

\begin{equation}\label{eq:gz_general}
  Y[k+2] = 
  \left( \begin{array}{cccc}
  -Y[k+1] & -Y[k] & U[k+1] & U[k]
  \end{array} \right)
  \left(\begin{array}{c}
    \alpha_1\\ \alpha_2 \\ \beta_1 \\ \beta_2
  \end{array}\right)
\end{equation}

A partir do qual podemos relacionar o sinal de saída com o sinal de entrada do sistema e obter os parâmetros $\alpha_1$,$\alpha_2$,$\beta_1$ e $\beta_2$ por regressão linear. Supondo 
$A = \left[\begin{array}{cccc}
 -Y[k+1] & -Y[k] & U[k+1] & U[k]
\end{array}\right]$.

Com auxílio do matlab foi obtido a seguinte função de transferência:

\begin{equation}
G(z) = \frac{Y(z)}{U(z)} = \frac{0.19422\,z-0.092392}{z^2-1.6576\,z+0.65762}
\end{equation}

%% TODO Colocar resultado regressão linear do matlab

\subsection{Controlador}

\subsubsection{Definição Requisitos}

Obviamente o melhor desempenho do controlador será algo dentro dos limites físicos do dispositivos envolvidos. No que inclui, a taxa de variação da posição angular não pode ser maior que $50Hz$, a precisão está bastante limitada as taxas de amostragem de 8 bits do Arduino e quantidade de buracos do disco do encoder. Tendo isto em mente foi definido como requisitos desejáveis para um controlar PID implementado em Arduino:

%%$$ Taxa de amortecimento


\subsubsection{Cálculo dos Parâmetros do Controlador}

%% Simulação

\subsubsection{Validação em Simulação}

%\subsubsection{Validação na planta}

\section{Conclusão}

A implementação de um sistema de controle de posição a partir de Arduino trouxe vários desafios. Preparar um hardware e garantir as mínimas condições de operação em tempo real para os controladores acabou tomando mais tempo que o previsto. Durante o processo alguns motores foram danificados e também houve perda de um encoder, porém tal representou uma excelente oportunidade de aprendizado tanto em como lidar com as características de cada dispositivo usado como também permitir uma compreensão melhor da teoria de controle.

Como abordagem principal, tanto na escolha do tema dos métodos utilizados foi buscado a todo momento a integração dos conhecimentos desenvolvidos da teoria de controle de sistema a tempo discreto em conjunto os campos de programação de microcontroladores, instrumentação digital e programação em tempo real. Esta abordagem trouxe seus ônus por acrescentar etapas no projeto além da simulação pura de um sistema físico idealizado, mas contribuiu bastante no melhor entendimento do impacto das características físicas e tipos de ruídos associados de cada parte no sistema de controle.

%% TODO Comentar sobre os resultados do controlador

% ------------------------------------------------------------------------------
\newpage
% Referências
\addcontentsline{toc}{section}{Referências} % Adiciona linha no índice
\bibliographystyle{abbrv} % Define Estilo e gera bibliografia
\bibliography{references} % Adiciona Arquivo com Referências

% Acrescentadas no arquivo references.bib
% para usa-las no texto basta usar \citep{}
% para citar sem usar no texto basta usar \nocite{}
%\nocite{sympy}
%\nocite{pythontex}
\nocite{matlabcontrol}
\nocite{matlabsymbolic}
\nocite{ogata2010modern}

% ------------------------------------------------------------------------------
%\newpage
\section*{Anexos}
\addcontentsline{toc}{section}{Anexos} % Adiciona linha no indice
%\subsection*{Python}

%Para os cálculos e demonstrações foi utilizado o pacote \textit{Python}\TeX\ \cite{pythontex} para o \LaTeX\ em conjunto da bibliteca \textit{sympy}\cite{sympy}. Segue o script completo em python:

%\inputminted[xleftmargin=15pt,linenos,frame=single,framesep=5pt,breaklines=true]{python}{../python/exsim6.py}

%\newpage
\subsection*{Matlab}

\subsubsection*{Parte 1}
Para o desenho dos gráficos e simulações foi utilizado o \textit{Matlab} em conjunto das toolbox \textit{Control System}\cite{matlabcontrol} e \textit{Symbolic Math}\cite{matlabsymbolic}. Segue o código referente usado

\inputminted[xleftmargin=15pt,linenos,frame=single,framesep=5pt,breaklines=true]{matlab}{../matlab/project.m}

\newpage
\subsection*{Arduino}
Dado as limitações do Arduino foram produzidos diferentes códigos como forma de avaliar o funcionamento de cada componente de hardware isoladamente.

\subsubsection*{Teste Encoder}
Para avaliar o funcionamento do encoder foi usado o seguinte código
\inputminted[xleftmargin=15pt,linenos,frame=single,framesep=5pt,breaklines=true]{c++}{../arduino/test_enconder/test_enconder.ino}

\newpage
\subsubsection*{Teste Motor}
Para avaliar o funcionamento do motor foi usado o seguinte código
\inputminted[xleftmargin=15pt,linenos,frame=single,framesep=5pt,breaklines=true]{c++}{../arduino/test_dcmotor/test_dcmotor.ino}

\newpage
\subsubsection*{Teste Motor}
O código usado para caracterização da zona morta do sistema e avaliação do funcionamento conjunto do motor com o encoder foi usado o seguinte:
\inputminted[xleftmargin=15pt,linenos,frame=single,framesep=5pt,breaklines=true]{c++}{../arduino/test_dcmotor_characterization/test_dcmotor_characterization.ino}

\newpage
\subsubsection*{Identificação da planta}
O código usado para identificação dos parâmetros do planta:
\inputminted[xleftmargin=15pt,linenos,frame=single,framesep=5pt,breaklines=true]{c++}{../arduino/test_dcmotor_pulses/test_dcmotor_pulses.ino}


\newpage
\subsubsection*{Controle}
Para implementação do controlador foi usado o seguinte código
\inputminted[xleftmargin=15pt,linenos,frame=single,framesep=5pt,breaklines=true]{c++}{../arduino/pid_control/pid_control.ino}

% https://www.usinainfo.com.br/drivers-para-motores/driver-ponte-h-ou-motor-de-passo-l298-2302.html
% ------------------------------------------------------------------------------
\end{document}
