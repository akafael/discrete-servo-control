% Pacotes e configurações padrão do estilo ``article''\
% -------------------------------------
\documentclass[a4paper,11pt]{article}
% Layout
% ------------------------------------------------------------------------------
\input{relat_layout.tex}

\usepackage{circuitikz}

\title{Projeto} % Define o título do Relatório
\author{Rafael Lima}

% Definições Auxiliares ( Macros próprias )
% ------------------------------------------------------------------------------
%\input{relat_aux.tex} % Arquivo com minhas macros
\newcommand{\npy}[1]{\sympy{round(#1,4)}}
% ----------------------------------~>ø<~---------------------------------------
\begin{document}
% Capa e Índice ----------------------------------------------------------------
%--------------------------------------------------- Capa --------------------------------------------
%\newpage
\begin{figure}[h!]
\centering
\includegraphics[scale=0.9]{img/simb_unb.png}
\label{fig:unb}
\end{figure}

\begin{center}
{\LARGE Universidade de Brasília}\\
Departamento de Engenharia Elétrica\\
Professor: Henrique Cezar Ferreira\\
Disciplina: Controle Digital\\
\end{center}


\vspace{0.18\textheight}

\begin{center}
    \Huge \textbf{\\\thetitle \\}
\end{center}

\vspace*{\fill} % Completa espaço em branco e empurra o resto para o final da página

% Tabela com os nome das pessoas do grupo

\begin{table}[H]
    \begin{tabular}{ll}
        % Nome      & Matrícula
        Rafael Lima & 10/0131093 \\
        Leonardo Cardoso Botelho & 11/0154151 \\
    \end{tabular}
\end{table}

\vspace{0.5cm}

\begin{center}
    \textbf{Brasília\\
    \the\year} % Coloca o Ano atual
\end{center}

\thispagestyle{empty} % Retira o cabeçalho e o rodapé da página

% ------------------------------------------------- Índice -------------------------------------------
\newpage
\tableofcontents
\newpage
% ----------------------------------------------------------------------------------------------------

 % Capa para UnB
% Conteúdo ---------------------------------------------------------------------

\section{Projeto de Controlador no Espaço de Estados}

% ------------------------------------------------------------------------------
\newpage
% Referências
\addcontentsline{toc}{section}{Referências} % Adiciona linha no indice
\bibliographystyle{abbrv} % Define Estilo e gera bibliografia
\bibliography{references} % Adiciona Arquivo com Referências

% Acrescentadas no arquivo references.bib
% para usa-las no texto basta usar \citep{}
% para citar sem usar no texto basta usar \nocite{}
%\nocite{sympy}
%\nocite{pythontex}
\nocite{matlabcontrol}
\nocite{matlabsymbolic}
\nocite{ogata2010modern}

% ------------------------------------------------------------------------------
%\newpage
%\section*{Anexos}
%\addcontentsline{toc}{section}{Anexos} % Adiciona linha no indice
%\subsection*{Python}

%Para os cálculos e demonstrações foi utilizado o pacote \textit{Python}\TeX\ \cite{pythontex} para o \LaTeX\ em conjunto da bibliteca \textit{sympy}\cite{sympy}. Segue o script completo em python:

%\inputminted[xleftmargin=15pt,linenos,frame=single,framesep=5pt,breaklines=true]{python}{../python/exsim6.py}

%\newpage
\subsection*{Matlab}

\subsubsection*{Parte 1}
Para o desenho dos gráficos e simulações foi utilizado o \textit{Matlab} em conjunto das toolbox \textit{Control System}\cite{matlabcontrol} e \textit{Symbolic Math}\cite{matlabsymbolic}. Segue o código referente usado

\inputminted[xleftmargin=15pt,linenos,frame=single,framesep=5pt,breaklines=true]{matlab}{../matlab/project.m}


% ------------------------------------------------------------------------------
\end{document}
