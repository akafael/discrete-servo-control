% Pacotes e configurações padrão do estilo ``article''\
% -------------------------------------
\documentclass[a4paper,11pt]{article}
% Layout
% --------------------------------------------------------------------------------
\input{relat_layout.tex}
%\usepackage{circuitikz}
\usepackage[makestderr]{pythontex}
%\restartpythontexsession{\thesection}

\newcommand{\tituloRelatorio}{Lista 1}
\title{\tituloRelatorio}
\hypersetup{pdftitle={\tituloRelatorio}}% title
\author{Rafael Lima}

% Definições Auxiliares
% --------------------------------------------------------------------------------
%\input{relat_aux.tex} % Arquivo com minhas macros
\renewcommand{\thesection}{Questão \arabic{section}}
\renewcommand{\thesubsection}{(\alph{subsection})}
\newcommand{\npy}[1]{\sympy{round(n#1,4)}}

% ----------------------------------~>ø<~---------------------------------------
\begin{document}
% Capa e Índice ---------------------------------------------------------------
%--------------------------------------------------- Capa --------------------------------------------
%\newpage
\begin{figure}[h!]
\centering
\includegraphics[scale=0.9]{img/simb_unb.png}
\label{fig:unb}
\end{figure}

\begin{center}
{\LARGE Universidade de Brasília}\\
Departamento de Engenharia Elétrica\\
Professor: Henrique Cezar Ferreira\\
Disciplina: Controle Digital\\
\end{center}


\vspace{0.18\textheight}

\begin{center}
    \Huge \textbf{\\\thetitle \\}
\end{center}

\vspace*{\fill} % Completa espaço em branco e empurra o resto para o final da página

% Tabela com os nome das pessoas do grupo

\begin{table}[H]
    \begin{tabular}{ll}
        % Nome      & Matrícula
        Rafael Lima & 10/0131093 \\
        Leonardo Cardoso Botelho & 11/0154151 \\
    \end{tabular}
\end{table}

\vspace{0.5cm}

\begin{center}
    \textbf{Brasília\\
    \the\year} % Coloca o Ano atual
\end{center}

\thispagestyle{empty} % Retira o cabeçalho e o rodapé da página

% ------------------------------------------------- Índice -------------------------------------------
\newpage
\tableofcontents
\newpage
% ----------------------------------------------------------------------------------------------------

 % Capa para UnB
% Conteúdo -------------------------------------------------------------------

\section{}

\begin{equation}
 \sympy{3} = \sympy{3/4}
\end{equation}

% ---------------------------------------------------------------------------------------

\bibliographystyle{abbrv}
\bibliography{references}
% Referências
% Acrescentadas no arquivo references.bib
% para usa-las no texto batsa usar \citep{}


% ---------------------------------------------------------------------------------------
\end{document}
